
% Definições para glossario

% ATENCAO o SHARELATEX GERA O GLOSSARIO/LISTAS DE SIGLAS SOMENTE UMA VEZ
% CASO SEJA FEITA ALGUMA ALTERAÇÃO NA LISTA DE SIGLAS OU GLOSSARIO É NECESSARIO UTILIZAR A OPÇÃO :
% "Clear Cached Files" DISPONIVEL NA VISUALIZAÇÃO DOS LOGS 
% ---
% https://www.sharelatex.com/learn/Glossaries


\newglossaryentry{pai}{
                name={pai},
                plural={pai},
                description={este é uma entrada pai, que possui outras
                subentradas.} }

 \newglossaryentry{componente}{
                name={componente},
                plural={componentes},
                parent=pai,
                description={descriação da entrada componente.} }
 
 \newglossaryentry{filho}{
                name={filho},
                plural={filhos},
                parent=pai,
                description={isto é uma entrada filha da entrada de nome
                \gls{pai}. Trata-se de uma entrada irmã da entrada
                \gls{componente}.} }

\newglossaryentry{tag} {
    name=tag,
    plural= {tags},
    description={Um marcador, a palavra em inglês significa etiqueta}
}
                
% Normalmente somente as palavras referenciadas são impressas no glossario, portanto é necessário referenciar utilizando \gls{identificação}                
