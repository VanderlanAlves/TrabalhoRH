
% ----------------------------------------------------------
% Introdução (exemplo de capítulo sem numeração, mas presente no Sumário)
% ----------------------------------------------------------
\chapter{Parte 1}

\section{Objetivos}
A primeira parte, consiste em apresentar a empresa Embraer, seu histórico,
ramo de atuação e os elementos públicos de seu planejamento estratégico (i.e., visão, missão e
valores da empresa). 

\section{Tema e Problematização}
A Embraer, cadastrada atualmente no CNPJ 07.689.002/0001-89, tem nome empresarial Embraer S.A. e está no setor aeroespacial desde agosto de 1969. Desde o seu surgimento, a empresa já produziu mais de 8 mil aeronaves e conquistou seu espaço em diversos lugares do mundo, tornando-se, assim, a quinta maior fabricante de jatos particulares, a terceira melhor produtora de jatos comerciais e a líder entre as empresas na produção de aviões comerciais que suportam possuem até 130 poltronas. Seu sucesso é tanto, que transporta mais de 140 milhões de passageiros todo ano, sendo que a cada 10 segundos uma aeronave Embraer inicia voo. A companhia também viu necessidade de criação de indústrias, escritórios e centro de distribuição que disponibilizassem peças em quase todos os continentes.



\section{Justificativa}
A inovação é fundamental para as Empresas atualmente. O livre mercado possibilita a livre concorrência entre Empresas e aquecimento Econômico. Com a maior demanda por pessoas, Empresas precisam se reinventar para captar os melhores talentos. 
Em um mercado aquecido, empresas investem pesado em mão de obra especializada e, também, em manter aquele talento engajado com os valores da Empresa, afinal, caso contrário, outra empresa oferce uma oportunidade mais atraente e a empresa acaba perdendo a sua mão especializada para outra companhia. Hodiernamente, não apenas a Embraer que apresenta uma proposta inovadora e engajadora para seus colaborades.



%\subsection{Objetivos}
%\lipsum[1]


%\section{Estrutura do Estudo}
%\lipsum[1]

