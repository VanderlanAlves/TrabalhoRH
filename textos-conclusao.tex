% ---
% Conclusão (outro exemplo de capítulo sem numeração e presente no sumário)
% ---
\chapter{Parte 3}

\section{Objetivo}
Analisar de forma crítica os mecanismos de contratação propostos pela Empresa Embraer, a fim de chegar em uma conclusão de possíveis
melhorias para tal processo empregado. 

\section{Resultado e Discussões}
O processo seletivo, de fato, consegue transmitir todos os requisitos que supostamente o entrevistador esteja procurando. Além de avaliar seu currículo, avalia também sua escrita, modo de como reage às situações proposta na dinâmica, de como se expressa e com a entrevista final que pode-se avaliar como o candidato se comporta em grupo
		Apesar de grande parte do funil do processo seletivo proposto pela Embraer ser online, a locomoção até a Empresa, por muita das vezes, acaba não sendo flexível para o candidato, dependendo da vaga proposta. Atualmente, é possível realizar um processo seletivo online, apenas com testes e entrevistas de forma virtual. No final do dia, a Empresa está procurando alguém que resolva o problema real apresentado e que possua a visão e valores que a empresa se baseia. Hoje em dia, existem diversos softwares que facilitam para o candidato e barateiam o custo da empresa em um processo seletivo. Nesse sentido, um futuro ideal e flexível para um processo seletivo de “sucesso” é selecionar os candidatos online e avaliá-lo pelo que entrega em um dia real de trabalho. O modelo de aplicar redações  pode ser considerado ultrapassado atualmente. Novos tempos, novos talentos. Para tal, é preciso meios mais produtivos e engajadores com o candidato. Em uma das etapas do processo, ser necessário ir presencialmente até a empresa acaba limitando a gama de talentos apenas em uma região que seja possível a locomoção. Por um outro lado, quando possuímos uma metodologia de contratação “without borders”, as chances de contratação e oportunidades são maiores para candidatos de qualquer lugar do Brasil e mundo participarem e fazer parte do time Embraer. 
	
O teste de Inglês está presente em grandes partes dos processos seletivos das grandes Empresas multinacionais, sendo, nesse sentido, uma etapa obrigatória. De fato, o inglês é fundamental em um mundo globalizado e dinâmico. Porém, testes de inglês tradicionais não apresentam muito do que o candidato sabe. Por exemplo, se for aplicado um teste da gramática do português e o candidato falhar, isso não significa necessariamente que ele não sabe português. A gramática é específica e não precisa-se saber ao pé da letra sobre ela para expressar-se e agregar valor para Empresa e gerar o lucro esperado por ela.Assim, contratar mais funcionários e seguir o fluxo natural da economia. Testes em um idioma deve ser feitos por meio de uma apresentação oral, por exemplo, apresentando uma ideia ou solução de algum produto. Dessa maneira, a empresa consegue avaliar, como um todo, a maneira  do candidato se comportar em uma situação real na companhia, caso tenha que apresentar algum produto. Mesmo que não seja este o cargo do candidato, é fundamental o desenvolvimento das tais Soft Skills que devem ser conhecidas tanto pelos recrutadores, quanto pelos candidatos. A habilidade de influenciar pessoas é tão fundamental quanto saber alguma Hard Skill!



\chapter{Conclusão}
De maneira geral, pode-se dizer que o processo seletivo da Embraer é bastante organizado e estruturado, dividindo-se, em: etapa presencial, para a entrega dos documentos requeridos; teste técnico; dinâmica de grupo e entrevista. Entretanto, esta linha de contratação está presente em muitas outras empresas, deixando evidente que a Embraer não inovou em novos meios de análise de candidatos além da forma que usa a inteligência artificial. Ainda sobre esse novo método, é importante salientar que a tecnologia deve ser implementada com cautela e somente após diversos testes, para não gerar problemas como a contratação de pessoas com características de muito parecidas, sejam elas Hard Skills ou Soft Skills, o que pode afetar a diversidade de empregados da empresa. Além disso, seria importante incluir pelo menos uma etapa em que um recrutador humano avalie o candidato, para que haja a certeza de que o colaborador esteja alinhado à empresa, sua visão e seus valores. Ademais, como citado anteriormente, a empresa pode abrir mais processos seletivos de forma online, para aumentar a gama de candidatos inscritos, o que aumenta a chance de contratação de pessoas ainda mais qualificadas para o cargo ofertado.
