% Para facilitar a manutenção é sempre melhore criar um arquivo por capitulo, para exemplo isso não é necessário 

%---------------------------------------------------------------------------------------
\chapter{Parte 1}

\section{História da Embraer}
Em 1950, a partir do suporte à aviação do governo de Getúlio Vargas, o ITA - Instituto Tecnológico de Aeronáutica, foi criado. Naquela época, o instituto tinha como objetivo a criação de um avião turbo propulsor para uso civil e militar. Assim, estruturou-se o projeto de criação do protótipo, que foi coordenado pelo engenheiro aeronáutico Ozires Silva. Depois que o avião concluiu com sucesso seus testes de voo, fez-se necessário criar uma fábrica que tornasse possível sua produção em larga escala. 
	Com esse objetivo, Ozires e sua equipe tentaram articular a criação da companhia com o  setor privado. Entretanto, o início da empresa se deu com capital misto, a partir da capitalização inicial do Estado. No dia 19 de Agosto de 1969, a Embraer (Empresa Brasileira de Aeronáutica S.A.) foi sancionada via Decreto de Lei. Já década de 70, a empresa ganhou proporções gigantes, dominando os céus dos quatro cantos do planeta e apresentando soluções para o segmento agrícola, comercial e executivo.
	Em 1994, após passar por dificuldades financeiras, a empresa foi totalmente privatizada e passou por uma grande reformulação, coordenada novamente por Ozires Silva. Em 2002, como forma de agradecimento à educação, criou o Colégio Engenheiro Juarez Wanderley, com o objetivo de dar oportunidade aos alunos em vulnerabilidade social da rede pública de ensino, oferecendo uma educação de ponta aos discentes. Desde então, a empresa continuou a produzir aeronaves de sucesso e segue até hoje riscando os céus do globo, se tornando uma das maiores empresas aeroespaciais de todo o mundo



\section{Perfil Institucional}

\subsection{Visão:}
	A Embraer foca em manter sua excelência e estabelecer-se no setor de aviação, sendo não somente um dos maiores destaques no seu ramo mercadológico, mas também liderando e sendo condecorada pela sua atuação empresarial na sociedade.

	
\subsection{Missão:}
	É extremamente importante para esta organização manter o nível de qualidade tecnológica de seus produtos de forma mais elevada possível, além de colocar um preço que seja ajustado de acordo com a concorrência do mercado.
Ademais, o padrão deve ser alto não somente para o produto, como também deve atender dentro do prazo e rapidamente às demandas dos clientes; sempre de modo confiável, criativo, eficiente, profissional e qualificado.


	
    \section{Valores:}

	\begin{description}
 	 \subsection{"Nossa Gente é o que nos faz VOAR":} 
            Nossa missão é levar toda a equipe de colaboradores para o próximo nível.
             Competentes, valorizados, realizados e comprometidos com a companhia em entregar o melhor para nossos clientes.
              Confiança é muito importante para manter todos alinhados com os valores adequados.
            \end{description}

            \begin{description}
                \subsection{"Existimos para servir nossos clientes"} 
                Sempre buscando a lealdade de nossos clientes com
                sua satisfação em nossos produtos e assim criando relações fortes e duradouras.
                \end{description}

                \begin{description}
                    \subsection{"Ética e Integridade está em tudo que fazemos":} 
                    Estamos sempre buscando o melhor para nossos almejados clientes e colab
                    oradores do time. Das atividades mais simples, as mais promissoras, o certo e a
                    ética sempre andam na frente de tudo.
                    \end{description}

                    \begin{description}
                        \subsection{"Buscamos a Excelência Empresarial"} 
                        Orientados pela simplicidade das coisas, agilidade, flexibilidade e segurança para agreg
                        ar o melhor para o cliente final. Ação empreendedora, voltado para o planejamento responsável e disciplinado
                        s para a execução de tarefas no melhor prazo possível.
                        \end{description}

                        \begin{description}
                            \subsection{"Ousadia e inovação são a nossa Marca"} 
                            Capacidade de inovação e aprendizado constante. Somos ousados e fascinados em mudar a realidade
                            interna e influenciar o nosso mercado. Coragem, Criatividade para superar os obstáculos diários.
                            \end{description}

                            \begin{description}
                                \subsection{"Atuação Global é a nossa Fronteira":} 
                                Pensamos além das barreiras impostas pela sociedade. A presença da nossa marca no quesito global,
                                é de suma importância para consolidação no nosso mercado e capacidade de abrangência sem limites.
                                \end{description}
          

\section{Responsabilidade Ambiental}
A Embraer traz para si a responsabilidade socioambiental, priorizando alguns ideais, como um gerenciamento ecológico, o respeito ao meio ambiente, a atenção ao ciclo de vida do produto etc. Desta forma, foi a primeira empresa de aviação no mundo a conseguir a norma ISO 14001 - que diz respeito ao Sistema de Gestão Ambiental. Nesta norma, são levados em conta tanto as questões ambientais influenciadas pela empresa, quanto aqueles que podem ser contidos por ela. Desde 2002 a organização foca no Sistema de Gestão Ambiental e a partir de 2009 tem, anualmente, seu inventário sobre emissão de gases do efeito estufa certificado pela International Organization for Standardization (ISO) já citada. Há também uma política de Meio Ambiente, Saúde e Segurança (MASS) no trabalho que foi implementada em todas as unidades da Embraer pelo mundo, onde as diretrizes são claras para que o trabalho seja desempenhado com qualidade, mas sem desrespeitar a natureza.
Como já citado, a Embraer não somente faz parte de acordos que prezam pela diminuição de CO2 e outros gases responsáveis pelo efeito estufa, como também trabalha em produtos que sejam mais sustentáveis. Para que eles possam se tornar realidade, existe o DIPAS (Desenvolvimento Integrado do Produto Ambientalmente Sustentável), que concentra sua energia em uma produção mais ecológica, trazendo alternativas tecnológicas para o ciclo de vida do produto. Esse programa mantém contato com áreas de estudo para a diminuição do lançamento de gás carbônico, consumo de combustível, ruídos e custo de manutenção.
Entretanto, seus esforços não se resumem somente à prudência referente aos gases do efeito estufa, já que também está envolvida em outras ações, como a atenção a seus efluentes e resíduos, a parceria com entidades interessadas em políticas públicas no setor de aviação e a adoção dos objetivos para o desenvolvimento sustentável da ONU.


\section{Responsabilidade Social}
Como citado anteriormente, a Embraer fundou o Colégio Engenheiro Juarez Wanderley em São José dos Campos no ano de 2002. Atualmente, ele oferece cerca de 960 vagas anuais para alunos em vulnerabilidade social. Além do ensino de ponta, gerenciado pelo Sistema Pitágoras, a escola fornece alimentação, uniforme, transporte e material didático incluso na bolsa de estudos. Ademais, para alunos que passaram em vestibulares fora da cidade, é oferecido bolsas de manutenção, que auxiliam no pagamento da moradia, do transporte e da alimentação dos ex-alunos



\chapter{Parte 2}
\section{Objetivo}
A segunda parte do trabalho deve apresentar como são os mecanismos de recrutamento, seleção e contratação de funcionários, considerando etapas, cronogramas e quais análises que são feitas. É esperado que você converse com pessoas de setores ligados a recrutamentos dentro da empresa, para que elas te passam tais informações. 



\section{Mecanismos de Recrutamento da EMBRAER}
“Não é possível pensar no futuro sem pensar nas pessoas”. Dessa forma, o capital mais importante para as empresas é, sem sombras de dúvida, as pessoas, o time, e o trabalho em equipe. Com isso, é de extrema importância que haja um processo de seleção eficaz, em que seja possível filtrar os melhores talentos e perfis para que encaixem melhor no objetivo da empresa naquele momento. 
	A Embraer, atualmente, tem passado por diversas mudanças no seu processo de seleção de novos talentos. Por isso, vale ressaltar que, para cada contratação, seja uma oportunidade de  estágio, Trainee, CLT, engenheiro de software etc, irá ocorrer uma abordagem específica para cada processo. O desafio é único na visão da empresa. Por isso, antes de concorrer para uma vaga na Embraer, é fundamental  estar de acordo com os requisitos propostos. Existem diversas oportunidades dentro da empresa para estágio,  destinados aos estudantes no penúltimo ou último ano de graduação. Logo, existe um fluxograma de contratação,  na forma de um passo a passo pré-determinado que a empresa segue até a entrevista final para a contratação de um estagiário. De fato, o processo acaba se tornando, como em grandes processos seletivos, um filtro  que elimina pessoas que possivelmente não estão necessariamente alinhadas com o objetivo da companhia ou  não cumprem com alguns requisitos essenciais da vaga. 


\chapter{Etapas do Processo Seletivo}
Primeiro, é fundamental estar matriculado em uma instituição de ensino superior. Logo, melhores informações são encontradas no próprio site da Empresa para candidatura. Após o envio do currículo.


\section{Encaminhamento via e-mail para o participante}
Será enviado um e-mail para possível realização de um teste de proficiência em inglês junto com a convocação para uma dinâmica e resultando na entrevista(presencial).

\section{ Parte Presencial do processo na sede da Empresa}
Assim, o candidato irá até a sede da empresa, levando os documentos necessários para o preenchimento de uma ficha com os dados pessoais. Para tal, é de fundamental ter em mãos um currículo atualizado, documentos em geral e, por último, um documento que comprove a matrícula na faculdade, constando o semestre que está cursando e a previsão de conclusão do curso.


\section{ Teste Técnico}
O primeiro desafio proposto pela empresa é avaliar o candidato por meio de uma Redação. Nessa etapa, a companhia tem como objetivo, além de observar no candidato a boa habilidade de escrita, possuir a  capacidade de articular ideias e pensar fora da caixa nas respostas propostas.

\section{ Dinâmica de grupo}
Após feito a {redação}, é feita a dinâmica de grupo, que é aparte fundamental do processo.A dinâmica consiste em fazer um desenho na cartolina ilustrando a trajetória do candidato até o momento presente e suas expectativas para o futuro. No final, o desenho é apresentado tanto para os outros participantes quanto para os analistas de Recursos Humanos

\section{ Entrevista com os analistas}
É realizado uma entrevista somente com os recrutadores para entender mais sobre o candidato e sua visão e colaboração para a companhia.

\section{Ligação ou notificação por email caso o candidato tenha passado}
O candidato é notificado por e-mail ou por telefone que passou no processo seletivo.

\section{ Etapas de exames médicos e a efetiva contratação do participante}
É realizado diversos exames médicos antes do participante começar o seu trabalho na empresa, mas, nesta etapa, já está praticamente tudo certo e em menos de alguns meses,  o candidato estará trabalhando.


\chapter{Inovação no Proceso de Recrutamento}
\section{Inteligência Artificial e Pessoas}
É de extrema importância a capacidade da companhia de inovar não somente no produto, mas em maneiras de melhorar o processo de contratação de pessoas. Dessa forma,
A Embraer em parceria com a Gupy, implementam uma inteligência artificial para auxiliar no processo seletivo de novos talentos para empresa. A primórdio, o intuito é automatizar o recrutamento de pessoas, ou seja, agilizando o processo tanto para empresa quanto para o candidato. Essa metodologia será implementada, em primeira instância, a vagas de estágio ofertadas pela empresa anualmente. Segundo o novo sistema, o candidato realizará um teste de lógica e inglês e todo processo de avaliação inicial será online. Após o candidato concluir as etapas referentes, irá receber um feedback automático referente a sua pontuação obtida. Dessa forma, segundo a Embraer, além de conseguir captar mais candidatos interessados em trabalhar na empresa, agiliza significativamente o processo de recrutamento e diminui gastos exagerados.
É imprescindível ressaltar o senso de inovação que a empresa passa para seus colaboradores e futuros jovens talentos que ajudam no crescimento da empresa. A Embraer acredita e investe arduamente no capital humano. No final do dia, na visão da companhia, são as pessoas que irão resolver os problemas e mover a companhia para o próximo nível. Quanto melhor o capital humano melhor será a visão da companhia, e, consequentemente, ganhos financeiros significativos no longo prazo.






